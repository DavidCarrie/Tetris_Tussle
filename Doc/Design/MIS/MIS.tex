\documentclass[12,english]{article}
\usepackage[letterpaper, portrait, margin=1in]{geometry}

\usepackage{amsmath}
\usepackage[T1]{fontenc}
\usepackage{babel}
\usepackage{textcomp}
\usepackage{titlesec}
\setcounter{secnumdepth}{4}
\usepackage{hyperref}
\usepackage{xcolor}
\usepackage{booktabs}
\usepackage{float}
\usepackage{placeins}
\usepackage{multirow}


\hypersetup{
    bookmarks=true,         % show bookmarks bar?
      colorlinks=true,       % false: boxed links; true: colored links
    linkcolor=black,          % color of internal links (change box color with linkbordercolor)
    citecolor=green,        % color of links to bibliography
    filecolor=magenta,      % color of file links
    urlcolor=cyan           % color of external links
}

\titleformat{\paragraph}
{\normalfont\normalsize\bfseries}{\theparagraph}{1em}{}
\titlespacing*{\paragraph}
{0pt}{3.25ex plus 1ex minus .2ex}{1.5ex plus .2ex}

\title{SE 3XA3: Module Interface Specification\\Tetris Tussle}

\author{Nicholas Lobo \\ lobon3 \\ 400179304 \and
	Matthew Paulin \\ paulinm \\ 400187147 \and
	David Carrie \\ carriedd \\ 000661652 \and
}


\date{}


\begin{document}
\maketitle
\newpage
\tableofcontents
\listoftables
\newpage

\section{Module Hierarchy}
\begin{table}[H]
\centering
\begin{tabular}{p{0.3\textwidth} p{0.6\textwidth}}
\toprule
\textbf{Level 1} & \textbf{Level 2}\\
\midrule

{Hardware-Hiding Module}& N/A\\
\midrule

\multirow{7}{0.3\textwidth}{Behaviour-Hiding Module} 
& User Inputs Module\\
& Main View Module\\
& Leaderboard Module\\
& Menu Module\\
& Singleplayer View Module\\
& Multiplayer View Module\\
& Singleplayer Module\\
& Multiplayer Module\\
& Player Module\\
& Tetromino Module\\
& Board Module\\

\midrule

Software Decision Module & Server Module\\ 

\bottomrule

\end{tabular}
\caption{Module Hierarchy}
\label{TblMH}
\end{table}

\section{MIS of Input Module}
		\subsection{Interface Syntax}
		\subsubsection{Exported Access Programs}
		\begin{tabular}[pos]{|c|c|c|c|}
			\hline
		\label{InputEAP}
			\textbf{Name}& \textbf{In} & \textbf{Out} & \textbf{Exceptions} \\ \hline
			onkeydown & KeyboardEvent Object & - & -\\ \hline
			
		\end{tabular}
		
		\subsection{Interface Semantics}
		\subsubsection{State Variables}
		keys: map (int $\rightarrow$ string) - A mapping of keyboard key codes to the corresponding game control input
		
		\subsubsection{Environmental Variables}
		N/A
		
		\subsubsection{Assumptions}
		N/A
		
		\subsubsection{Access Program Semantics}
		onkeydown(event):\\
		
		Input: Object event, a JavaScript Keyboard Event Object representing a user's keyboard input\\
		
		Transition: Translate the key code into a control input and pass the value to the Player module through a call of the keyPress function \\
		
		Output: None\\
		
		Exceptions: None\\
\section{MIS of Main View Module}
		\subsection{Interface Syntax}
		\subsubsection{Exported Access Programs}
		\begin{tabular}[pos]{|c|c|c|c|}
			
			\hline
			\label{MainviewEAP}
			\textbf{Name}& \textbf{In} & \textbf{Out} & \textbf{Exceptions} \\ \hline
			mouseClicked & MouseEvent Object & GUI & -\\ \hline

			
		\end{tabular}
		
		\subsection{Interface Semantics}
		\subsubsection{State Variables}
		viewState: string - an indication of the current view to be displayed
		\subsubsection{Environmental Variables}
		N/A
		\subsubsection{Assumptions}
		N/A
		
		\subsubsection{Access Program Semantics}
		
		mouseClicked(event):
		
		Input: Object event, a JavaScript Mouse Event Object representing a user's mouse input\\
		
		Transition: Translate the mouse input into a control input and adjusts the viewState variable\\
		
		Output: The visual elements to be displayed\\
		
		Exceptions: None\\
\section{MIS of Leaderboard Module}
	\subsection{Interface Syntax}
		\subsubsection{Exported Access Programs}
		\begin{table}[!htbp]
		\begin{tabular}{|c|c|c|c|}
			\hline
			\label{SingleviewEAP}
			Name & In & Out & Exceptions \\ \hline
			updateLeaderboard & string, int & GUI & - \\ \hline
		\end{tabular}
	\end{table}
		
	\subsection{Interface Semantics}
		\subsubsection{State Variables}
		highScores: List : int
		
		\subsubsection{Environmental Variables}
        N/A
		\subsubsection{Assumptions}
        N/A
		\subsubsection{Access Program Semantics}

        updateLeaderboard(name,score):

		Input: name: string - the name of the player, score: int - the point total for a completed game
		
		Transition: updates the menu view module and the highScores state variable
		
		Output: A list of high scores shown on screen
		
		Exception - None\\
\section{MIS of Menu Module}
	\subsection{Interface Syntax}
		\subsubsection{Exported Access Programs}
		
	\begin{tabular}[pos]{|c|c|c|c|}
	\hline
	\textbf{Name}& \textbf{In} & \textbf{Out} & \textbf{Exceptions} \\ 
	\hline
	changeView & string & GUI & -\\ 
	\hline
	\end{tabular}		
		
	\subsection{Interface Semantics}
		\subsubsection{State Variables}
		currentView: string- represents the current view being displayed
		
		\subsubsection{Environmental Variables}
		N/A
		\subsubsection{Assumptions}
        N/A
		\subsubsection{Access Program Semantics}
		changeView(currView):

		Input: string value to change the current view being displayed
		
		Transition: sets currentView to the value of currView
		
		Exception: none\\
		\\


\newpage
\section{MIS of Singleplayer View Module}
	\subsection{Interface Syntax}
		\subsubsection{Exported Access Programs}
		\begin{table}[!htbp]
				\begin{tabular}{|c|c|c|c|}
					\hline
					\textbf{Name}& \textbf{In} & \textbf{Out} & \textbf{Exceptions} \\ \hline
					setup &  - & GUI & - \\ \hline
				    draw &  - & - & - \\ \hline
				    display &  Player & GUI & - \\ \hline
				    windowResized & - & GUI & - \\ \hline
				    drawBlock & integer,integer,integer,integer,string,integer, & GUI & - \\ \hline
				    drawBorder & integer, List(integer) & GUI & - \\ \hline
				    drawGrid & integer, List(integer) & GUI & - \\ \hline
				    drawBoard & integer, List(integer) & GUI & - \\ \hline
				    drawPlaced & integer, List(integer), Board & GUI & - \\ \hline
				    drawCurrent & integer, List(integer), List(integer), integer, string, Tetromino, integer & GUI & - \\ \hline
				    
				\end{tabular}
			\end{table}
	\subsection{Interface Semantics}
		\subsubsection{State Variables}
		
		COLORS: List: String - list of different colour hex codes that represents the colours of the Tetromino\\
		
		
		\subsubsection{Environmental Variables}
		
		N/A 
		
		\subsubsection{Assumptions}
		
		N/A

		\subsubsection{Access Program Semantics}
		%
% 		getRecordingStatus():
		
% 		Input: none
		
% 		Transition: accesses variable isRecording and retrieves the value
		
% 		Output: returns value of variable isRecording
		
% 		Exception: none\\
% 		\\
		%
		
	  setup():
		
	  Input: 
		
	  Transition: displays p5.js canvas on screen
	  
	  Output: GUI
	  
	  Exception - none\\ 
	  \\
	  draw():
		
	  Input: 
		
	  Transition: needed for the processing library to function correctly
	  
	  Output: 
	  
	  Exception - none\\ 
	  \\
	  display(player):
		
	  Input: player - player object
		
	  Transition: uses other methods to display game state on screen
	  
	  Output: 
	  
	  Exception - none\\
	  \\
	  windowResized():
		
	  Input: player -
		
	  Transition: Updates size of canvas to match size of browser window
	  
	  Output: GUI
	  
	  Exception - none\\ 	  	  
	  \\
	  drawBlock(x,y,outline,strokecolor,color,unit):
		
	  Input: x - x position of a block, y - y position of block, outline - stroke weight of the block outline,
	  color - color of the block, unit- size of the block
		
	  Transition: fills in a grid square
	  
	  Output: GUI
	  
	  Exception - none\\ 	  	  
	  \\
	  drawBorder(units, topleft):
		
	  Input: units - size of a grid square, topleft- x,y position of the board
		
	  Transition: displays border for the board
	  
	  Output: GUI
	  
	  Exception - none\\ 	  	  
	  \\
	drawGrid(units, topleft):
	
	  Input: units - size of a grid square, topleft- x,y position of the board
		
	  Transition: displays the grid lines for the board
	  
	  Output: GUI
	  
	  Exception - none\\ 
	  \\
	drawBoard(units, topleft):
	
	  Input: units - size of a grid square, topleft- x,y position of the board
		
	  Transition: calls the drawGrid and drawBoard function
	  
	  Output: GUI
	  
	  Exception - none\\ 	  
	  \\
	drawPlaced(units, topleft, board):
	
	  Input: units - size of a grid square, topleft- x,y position of the board, board - board object
		
	  Transition: draws the Tetrominos placed on the board
	  
	  Output: GUI
	  
	  Exception - none\\ 	  	  
\\
	drawCurrent(unit, position, topleft, shadow. color, shape ):
	
	  Input: units - size of a grid square, position - position of the Tetromino, topleft- x,y position of the board, 
	  shadow - The vertical position of the shadow piece, color - the color of the Tetromino, shape - an integer giving the binary representation of the Tetromino
		
	  Transition: draws the Tetrominos placed on the board
	  
	  Output: GUI
	  
	  Exception - none 
	  
	  
	  \newpage
\section{MIS of Multiplayer View Module}
	\subsection{Interface Syntax}
		\subsubsection{Exported Access Programs}
		\begin{table}[!htbp]
				\begin{tabular}{|c|c|c|c|}
					\hline
					\label{MultiviewEAP}
					\textbf{Name}& \textbf{In} & \textbf{Out} & \textbf{Exceptions} \\ \hline
					setup &  - & GUI & - \\ \hline
				    draw &  - & - & - \\ \hline
				    display &  Player, Player & GUI & - \\ \hline
				    windowResized & - & GUI & - \\ \hline
				    drawBlock & integer,integer,integer,integer,string,integer, & GUI & - \\ \hline
				    drawBorder & integer, List(integer) & GUI & - \\ \hline
				    drawGrid & integer, List(integer) & GUI & - \\ \hline
				    drawBoard & integer, List(integer) & GUI & - \\ \hline
				    drawPlaced & integer, List(integer), Board & GUI & - \\ \hline
				    drawCurrent & integer, List(integer), List(integer),integer, string, Tetromino, integer & GUI & - \\ \hline
				    
				\end{tabular}
			\end{table}
	\subsection{Interface Semantics}
		\subsubsection{State Variables}
		
		COLORS: List: String - list of different colour hex codes that represents the colours of the Tetromino\\
		
		
		\subsubsection{Environmental Variables}
		
		N/A 
		
		\subsubsection{Assumptions}
		
		N/A

		\subsubsection{Access Program Semantics}
		
	  setup():
		
	  Input: 
		
	  Transition: displays p5.js canvas on screen
	  
	  Output: GUI
	  
	  Exception - none\\ 
\\
	  draw():
		
	  Input: 
		
	  Transition: needed for the processing library to function correctly
	  
	  Output: 
	  
	  Exception - none\\ 
	  \\
	  display(player,player2):
		
	  Input: player - player object
		
	  Transition: uses other methods to display game state on screen for two players
	  
	  Output: 
	  
	  Exception - none\\ 
	  \\
	  windowResized():
		
	  Input: player -
		
	  Transition: Updates size of canvas to match size of browser window
	  
	  Output: GUI
	  
	  Exception - none\\ 	  	  
	  \\
	  drawBlock(x,y,outline,strokecolor,color,unit):
		
	  Input: x - x position of a block, y - y position of block, outline - stroke weight of the block outline,
	  color - color of the block, unit- size of the block
		
	  Transition: fills in a grid square
	  
	  Output: GUI
	  
	  Exception - none\\ 	  	  
	  \\
	  drawBorder(units, topleft):
		
	  Input: units - size of a grid square, topleft- x,y position of the board
		
	  Transition: displays border for the board
	  
	  Output: GUI
	  
	  Exception - none\\ 	  	  
	  \\
	drawGrid(units, topleft):
	
	  Input: units - size of a grid square, topleft- x,y position of the board
		
	  Transition: displays the grid lines for the board
	  
	  Output: GUI
	  
	  Exception - none\\ 
	  \\
	drawBoard(units, topleft):
	
	  Input: units - size of a grid square, topleft- x,y position of the board
		
	  Transition: calls the drawGrid and drawBoard function
	  
	  Output: GUI
	  
	  Exception - none\\ 	  
	  \\
	drawPlaced(units, topleft, board):
	
	  Input: units - size of a grid square, topleft- x,y position of the board, board - board object
		
	  Transition: draws the Tetrominos placed on the board
	  
	  Output: GUI
	  
	  Exception - none\\ 	  	  
\\
	drawCurrent(unit, position, topleft, shadow. color, shape ):
	
	  Input: units - size of a grid square, position - position of the Tetromino, topleft- x,y position of the board, 
	  shadow - The vertical position of the shadow piece, color - the color of the Tetromino, shape - an integer giving the binary representation of the Tetromino
		
	  Transition: draws the Tetrominos placed on the board
	  
	  Output: GUI
	  
	  Exception - none
\section{MIS of Tetromino Module}
		\subsection{Interface Syntax}
			\subsubsection{Exported Access Programs}
				\begin{tabular}[pos]{|c|c|c|c|}
					
					\hline
					\label{TetrominoEAP}
					\textbf{Name}& \textbf{In} & \textbf{Out} & \textbf{Exceptions} \\ \hline
					Tetromino &  - & Tetromino Object & -\\ \hline
                    getPosition &  - & List(integer) & -\\ \hline
                    getShape &  - & integer & -\\ \hline
                    getState &  - &  List(integer)& -\\ \hline
                    getRotation &  - & integer & -\\ \hline
                    rotate &  - & -  & -\\ \hline
					getShadow &  - & integer & -\\ \hline
				    setShadow &  integer & -  & -\\ \hline
			        gameTick &  - & - & - \\ \hline
			        move & string & - & - \\ \hline
				    drop & integer & - & - \\ \hline
 				\end{tabular}
				
		\subsection{Interface Semantics}
			\subsubsection{State Variables}
			shape: integer - represents corresponding index in SHAPES\\  
			state: List(integer) - lists of binary encoded all rotations of each shape\\
			rotation: integer - specifies the rotation of shape and corresponds to index in shape array\\
			shadow: integer - represents the vertical position of the shadow\\
			position: List(integer) - x and y position of the Tetromino\\
			
			
			
			\subsubsection{Environmental Variables}
			SHAPES: List(List(integer)) - stores the binary representations of each Tetromino for each rotation position
			
			\subsubsection{Assumptions}
			Variables should be set before trying to access them
			
			\subsubsection{Access Program Semantics}
			
			Tetromino():
			
			Input: 
			
			Transition: creates a random Tetromino object in its default position
			
			Output: Tetromino Object
			
			Exception: none\\
			\\
			getPosition():
			
			Input: none
			
			Transition: none
			
			Output: a list of integers containing the x and y position of the Tetromino
			
			Exception: none\\
	        \\			
			getShape():
			
			Input: none
			
			Transition: none
			
			Output: An integer representing the Tetromino type's corresponding location in the SHAPES list
			
			Exception: none\\
			\\							
			getState():
			
			Input: none
			
			Transition: none
			
			Output: A list of integers containing the binary encoded versions of the Tetromino in all its possible rotations
			
			Exception: none\\
			\\		
			getRotation():
			
			Input: none
			
			Transition: none
			
			Output: An integer representing the current rotation of the Tetromino
			
			Exception: none\\
			\\
			rotate():
			
			Input: none
			
			Transition: rotates the Tetromino 90 degrees
			
			Output: none
			
			Exception: none\\
			\\		
			getShadow():
			
			Input: none
			
			Transition: none
			
			Output: An integer representing the vertical position of the Tetromino's shadow
			
			Exception: none\\
			\\	
			setShadow(shadow):
			
			Input: shadow - An integer representing the vertical position of the Tetromino's shadow
			
			Transition: update the Tetromino's shadow
			
			Output: none
			
			Exception: none\\
			\\	
			gameTick():
			
			Input: none
			
			Transition: move the Tetromino downwards 
			
			Output: none
			
			Exception: none\\
			\\	
			move(direction):
			
			Input: direction - A string representing the direction of movement
			
			Transition: update the Tetromino's position
			
			Output: none
			
			Exception: none\\
			\\	
			drop(y):
			
			Input: y - an integer representing the vertical amount to move
			
			Transition: update the Tetromino's position
			
			Output: none
			
			Exception: none\\
			\\	
\section{MIS of Board Module}
		\subsection{Interface Syntax}
			\subsubsection{Exported Access Programs}
				\begin{tabular}[pos]{|c|c|c|c|}
					
					\hline
					\label{BoardEAP}
					\textbf{Name}& \textbf{In} & \textbf{Out} & \textbf{Exceptions} \\ \hline
					Board &  - & Board Object & -\\ \hline
					getElems &  - & List(List(Cell)) & -\\ \hline
					addToBoard &  Tetromino & - & -\\ \hline
                    checkCollisions &  Tetromino, List(integer), integer & boolean & -\\ \hline
                    hardDrop &  Tetromino & integer & -\\ \hline
                    clearLine &  -  & - & -\\ \hline
 				\end{tabular}
				
		\subsection{Interface Semantics}
			\subsubsection{State Variables}
			elems: List(List(cells)) - represents one grid unit of the board, if its filled and the current color of the grid unit. 
			
			
			
			\subsubsection{Environmental Variables}
			ROWS: integer - number of rows on the board
			COLS: integer - number of columns on the board
			
			\subsubsection{Assumptions}
			None
			
			\subsubsection{Access Program Semantics}
			
			Board():
			
			Input: 
			
			Transition: creates an empty board object
		
			Output: Board Object
			
			Exception: none\\
			\\
			getElems():
			
			Input: none
			
			Transition: none
			
			Output: Elems
			
			Exception: none\\
	        \\			
			addToBoard(Tetromino):
			
			Input: Tetromino object
			
			Transition: fills in cells containing the tetromino on the board
			
			Output: none 
			
			Exception: none\\
			\\							
			checkCollision(Tetromino, direction, rotation):
			
			Input: Tetromino Object, direction - List(integer) representing horizontal and vertical direction of the Tetromino, rotation - integer representing the way the Tetromino will rotate 
			
			Transition: Checks collisions between Tetromino and the board
			
			Output: Boolean representing if a collision occurred
			
			Exception: none\\
			\\		
			
			hardDrop(Tetromino):
			
			Input: Tetromino - a Tetromino object 
			
			Transition: Lowest position the Tetromino can move to without colliding with the board
			
			Output: An integer representing the vertical position of the lowest point on the board
			
			Exception: none\\
			\\
			clearLine():
			
			Input: none
			
			Transition: Delete filled in rows on the board and shifts all filled pieces on the board down one unit.
			
			Output: none
			
			Exception: none\\
			\\		
\section{MIS of Player Module}
		\subsection{Interface Syntax}
			\subsubsection{Exported Access Programs}
				\begin{tabular}[pos]{|c|c|c|c|}
					
					\hline
					%	\label
					\textbf{Name}& \textbf{In} & \textbf{Out} & \textbf{Exceptions} \\ \hline
					Player & integer, Board, Tetromino, boolean   & Player Object & -\\ \hline
					getId &  - & integer & -\\ \hline
					getBoard &  - & Board & -\\ \hline
					getTetromino &  - & Tetromino & -\\ \hline
					isPaused &  - & boolean & -\\ \hline
					setBoard &  Board & - & -\\ \hline
                    setTetromino &  Tetromino & - & -\\ \hline
                    togglePaused &  - & - & -\\ \hline
                    
 				\end{tabular}
				\label{PlayerEAP}
		\subsection{Interface Semantics}
			\subsubsection{State Variables}
			id: integer - represents a unique player identifier\\ 
			board: Board - current board with respect to player id\\
			tetromino: Tetromino - current Tetromino with respect to player id \\
			paused: boolean - checks whether the game is paused
			
			
			\subsubsection{Environmental Variables}

			
			\subsubsection{Assumptions}
			None
			
			\subsubsection{Access Program Semantics}
			
			Player(id, board, tetromino, paused):
			Input: id - integer that represents a unique player identifier, board - Board - current board with respect to player id,
			tetromino - current Tetromino with respect to player id, paused - boolean that checks whether the game is paused
			
			Transition: creates a Player  object
		
			Output: Player Object
			
			Exception: none\\
			\\
			getId():
			
			Input: none
			
			Transition: gets the unique id of the player object 
			
			Output: id 
			
			Exception: none\\
	        \\			
			getBoard():
			
			Input: none
			
			Transition:  gets the unique Board object of the player object 
			
			Output: Board object
			
			Exception: none\\
			\\			
			getTetromino():
			
			Input: none
			
			Transition:  gets the unique Tetromino object of the player object 
			
			Output: Tetromino object
			
			Exception: none\\
			\\					
			isPaused():
			
			Input: none
			
			Transition:  gets the value of the isPaused state variable for the player object
			
			Output: boolean
			
			Exception: none\\
			\\
			setTetromino():
			
			Input: Tetromino
			
			Transition:  gets the unique Tetromino object of the player object 
			
			Output: Tetromino object
			
			Exception: none\\
			\\
			setBoard():
			
			Input: Board
			
			Transition:  sets the attributes of the board for the player object 
			
			Output: none
			
			Exception: none\\
			\\
			togglePause():
			
			Input: none
			
			Transition:  changes the value of the isPaused state variable
			
			Output: none
			
			Exception: none\\
			\\			
			
			
			
			
			
			
			
			
			
			
			
\section{MIS of Singleplayer Module}
		\subsection{Interface Syntax}
			\subsubsection{Exported Access Programs}
				\begin{tabular}[pos]{|c|c|c|c|}
					
					\hline
					%	\label
					\textbf{Name}& \textbf{In} & \textbf{Out} & \textbf{Exceptions} \\ \hline
					tick &  - & GUI & -\\ \hline
					getState &  - & Player Object & -\\ \hline
					display &  Player Object & GUI & -\\ \hline
					newGame &  - & GUI & -\\ \hline
					keyPress & string & GUI & -\\ \hline
 				\end{tabular}
				\label{SingleplayerEAP}
		\subsection{Interface Semantics}
			\subsubsection{State Variables}
			player: a Player object
			\subsubsection{Environmental Variables}
			\subsubsection{Assumptions}
			None
			\subsubsection{Access Program Semantics}
			
			tick():
			Input: none
			
			Transition: Advances the game at a set interval and updates the Player accordingly
		
			Output: updates GUI
			
			Exception: none\\
			\\	
			getState():
			Input: none
			
			Transition: Gets the current state of the game and displays that state
		
			Output: a player
			
			Exception: none\\
			\\	
			display(player):
			Input: player - a Player object
			
			Transition: Sends an update to the client to update graphical elements
		
			Output: updates GUI
			
			Exception: none\\
			\\	
			newGame():
			Input: none
			
			Transition: Creates a new game and updates the player accordingly
		
			Output: updates GUI to reflect the new game
			
			Exception: none\\
			\\	
			keyPress(key):
			Input: key - a string representation of the key being pressed
			
			Transition: Handles the movement and updates the player accordingly
		
			Output: updates GUI to reflect the requested movement
			
			Exception: none\\
			\\
\section{MIS of Multiplayer Module}
		\subsection{Interface Syntax}
			\subsubsection{Exported Access Programs}
				\begin{tabular}[pos]{|c|c|c|c|}
					
					\hline
					%	\label
					\textbf{Name}& \textbf{In} & \textbf{Out} & \textbf{Exceptions} \\ \hline
					tick &  - & GUI & -\\ \hline
					getState &  - &  List(Player Object) & -\\ \hline
					display &  List(Player Object) & GUI & -\\ \hline
					newGame &  - & GUI & -\\ \hline
					keyPress & string & GUI & -\\ \hline
 				\end{tabular}
 				\label{MultiplayerEAP}
				
		\subsection{Interface Semantics}
			\subsubsection{State Variables}
			players: a list of Player objects playing a multiplayer game against each other
			\subsubsection{Environmental Variables}
			\subsubsection{Assumptions}
			None
			\subsubsection{Access Program Semantics}
			
			tick():
			Input: none
			
			Transition: Advances the game at a set interval and updates the players accordingly
		
			Output: updates GUI
			
			Exception: none\\
			\\	
			getState():
			Input: none
			
			Transition: Gets the current state of the game and displays that state
		
			Output: players - the list of player objects playing the current game
			
			Exception: none\\
			\\	
			display(players):
			Input: players - a list of Player objects
			
			Transition: Sends an update to the client to update graphical elements
		
			Output: updates GUI
			
			Exception: none\\
			\\	
			newGame():
			Input: none
			
			Transition: Creates a new game and updates the players accordingly
		
			Output: updates GUI to reflect the new game
			
			Exception: none\\
			\\	
			keyPress():
			Input: key - a string representation of the key being pressed
			
			Transition: Handles the movement and updates the players accordingly
		
			Output: updates GUI to reflect the movement
			
			Exception: none\\
			\\
			
			
\section{MIS of Server Module}
	\subsection{Interface Syntax}
		\subsubsection{Exported Access Programs}
		
	\begin{tabular}[pos]{|c|c|c|c|}
	\hline
	\textbf{Name}& \textbf{In} & \textbf{Out} & \textbf{Exceptions} \\ 
	\hline
	setInterval & integer & -  & -\\ \hline
	newConnection & Socket Object & -  & -\\ \hline
	update & Player Object & -  & -\\ \hline
	disconnect & Socket Object & -  & -\\ \hline
	heartbeat & - & GUI  & -\\ \hline
	\end{tabular}		
	\label{ServerEAP}	
	\subsection{Interface Semantics}
		\subsubsection{State Variables}
		port: int - an integer representing the server's port\\
		server: express server - the server\\
		players: List(Sockets) - a list of connected clients
		
		\subsubsection{Environmental Variables}
		N/A
		\subsubsection{Assumptions}
        N/A
		\subsubsection{Access Program Semantics}
		setInterval(interval):

		Input: interval - integer that represents the interval by which the game should be updated on
		
		Transition: calls the heartbeat function at this new interval
		
		Output: none
		
		Exception: none\\
		\\
		newConnection():

		Input: Socket object  
		
		Transition: Adds the socket to the connected clients and creates a new game
		
		Output: none
		
		Exception: none\\
		\\		
		update(player):

		Input: player - Player object to be updated
		
		Transition: update the player to reflect the latest game movements
		
		Output: none
		
		Exception: none\\
		\\		
		disconnect():

		Input: Socket integer
		
		Transition: remove the socket from the list of active connections
		
		Output: none
		
		Exception: none\\
		\\
		heartbeat():

		Input: none
		
		Transition: update all connected clients with the latest Player data to be displayed
		
		Output: none
		
		Exception: none\\
		\\		
\textbf{}
\section{Major Revision History}
March 14, 2021 - Rough draft\\
March 16, 2021 - Revised draft \\
March 18, 2021 - Revision 0 complete

			
\end{document}
